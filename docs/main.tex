%%%%%%%%%%%%%%%%%%%%%%%%%%%%%%%%%%%%%%%%%%%%%%%%%%%%%%%%%%%%%%%%%%%
%
%    MINIJET event generator paper.
%
%     by A.Miyamoto (KEK) and H. Hayashii (NWU)
%
%%%%%%%%%%%%%%%%%%%%%%%%%%%%%%%%%%%%%%%%%%%%%%%%%%%%%%%%%%%%%%%%%%%

\documentstyle [12pt]{article}
\input setup.tex

\begin{document}


%%%%%%%%%%%%%%%%%%%
%
%  The title of the paper
%
%%%%%%%%%%%%%%%%%%%

%\centerline{Draft:Not for distribution}
%\centerline{version: \today}

\begin{center}
{\bf \vglue 10pt 
   A MONTE CALRO PROGRAM TO GENERATE  \\                      
               \vglue 3pt                                                       
   MINI-JET EVENTS IN TWO-PHOTON PROCESS \\}
\end{center}
\vglue 5pt

\begin{center}
Akiya Miyamoto\footnote{e-mail address: miyamoto@kekux1.kek.jp} \\
{\it KEK, National Laboratory for High Energy Physics\\
1-1 Oho, Tsukuba, Ibaraki 305, Japan} \\
and \\
Hisaki Hayashii\footnote{e-mail address: thkh@jpnkektr.bitnet} \\
{\it Department of Physics, Nara Woman's University, 
Nara 630, Japan}
\end{center}

\vglue 0.3cm
\begin{center}{\bf ABSTRACT} \end{center}
An event generator for resolved photon processes in photon-photon
collisions in $e^+e^-$ reactions
is developed.  The program can generate direct, once-resolved
and twice-resolved processes separately or simultaneously according
to user switches.

%\begin{center}
%{\rightskip=3pc
% \leftskip=3pc
% \tenrm\baselineskip=12pt }
%\end{center}

\baselineskip=14pt


%%%%%%%%%%%%%%%%%%%%%%%%%%%%%%%%%%%%%%%%%%%%%%%%%%%%%%%%%%%%
%
%  Chapter 1.
%
%%%%%%%%%%%%%%%%%%%%%%%%%%%%%%%%%%%%%%%%%%%%%%%%%%%%%%%%%%%%

\vglue 14pt
\centerline{ {\bf PROGRAM SUMMARY}}
\vglue 14pt

{\tenrm
\noindent {\tenit Title of program : MINIJET}
\vglue 12pt

\noindent {\tenit Program obtainable from}: CPC Program Library,
Queen's University of Belfast, N. Ireland, and the authors
upon the request by e-mail.}

\vglue 12pt

\noindent {\tenit Computer for which the program is designed and others on
which it is operable}: FACOM and others with FORTRAN 77 compiler.
\vglue 12pt

\noindent {\tenit Computer}: FACOM-M1800; {\it Installation}: National
Laboratory for High Energy Physics (KEK), Tsukuba, Ibaraki, Japan.
\vglue 12pt

\noindent {\tenit Operating system}: OSIV MSP-E20, OSIV EX.
\vglue 12pt

\noindent {\tenit Program language used}: FORTRAN77
\vglue 12pt

\noindent {\tenit High speed storage required}: 1300 k words.
\vglue 12pt

\noindent {\tenit No. of bits in a word}: 32
\vglue 12pt

\noindent {\tenit Peripherals used}: terminal or card reader for input, disk
and printer for output.
\vglue 12pt

\noindent {\tenit No. of lines in combined program and test deck}: 2629
\vglue 12pt

\noindent {\tenit Keywords}: Monte Carlo simulation, event generation,
mini-jet, two-photon process, resolved photon process.
\vglue 12pt
	
\noindent {\tenit Nature of physical problem}: \\
In high energy two-photon processes in $e^+e^-$ reactions,
the hadronic components in the photon contribute significantly
to the process. In particular, 
jet production in (almost real) photon-photon collisions offers
an opportunity to test pertuabative QCD.
In addition, the understanding of the process
is indispensable for designing future
high energy $e^+e^-$ linear colliders
%
%best process to test our knowledge concerning the parton distribution
%in the photon.
%almost real photon resolved into quarks and gluons.
%Collisions among them, resolved photon process, 
%constitute the significant part of high $p_t$ jet productions.  
%This process provides us a filed to test perturbative QCD
%Since the collisions between the partons (
%{\it i.e.} quarks and gluons ) inside the photon contribute to 
%the significant part of high $p_t$ jet production,
%the understanding is crucial 
%for designing future
%high energy $e^+e^-$ linear colliders.

\vglue 12pt

\noindent {\tenit Method of solution}: \\ 
The cross sections of the resolved photon processes
are calculated in the parton level 
by a convolution of photon luminosity function,
photon structure functions and sub-process cross sections.
Based on the calculation, high $p_t$ partons
and remnant partons 
accompanied with the resolved photon processes are generated.
They are hadronized subsequently, while neglecting scattered
$e^\pm$.
\vglue 12pt

\noindent {\tenit Typical running time}: \\
The running times depends upon physical parameters.
In the case of a default parameter set, it
takes 487 sec for calculating
cross section on FACOM-M1800 computer,
while 2.85 sec to generate 1k events.
\vglue 12pt


%%%%%%%%%%%%%%%%%%%%%%%%%%%%%%%%%%%%%%%%%%%%%%
%
%  Long Writeup
%
%%%%%%%%%%%%%%%%%%%%%%%%%%%%%%%%%%%%%%%%%%%%%%

\centerline{\bf LONG WRITE-UP}

\section{Introduction}

The program package presented here is to calculate the
cross section of jet productions by a
collision of almost real $\gamma\gamma$ 
in $e^+e^-$ collisions 
and generate such events for Monte Calro studies.
There are three different classes of diagrams
which contribute to the mini-jet production in 
$\gamma\gamma$ collisions,
as shown in Fig.~\ref{DIAGRAM}.
We call each diagram as direct(Fig.~\ref{DIAGRAM}a),
once-resolved(Fig.~\ref{DIAGRAM}b) and 
twice-resolved(Fig.~\ref{DIAGRAM}c), respectively.
Those three processes contribute to the jet production in
$\gamma\gamma$ collision in the same order,
%In a collision between almost real $\gamma\gamma$,
%resolved photon process constitute the significant part
%of the cross section of high $p_t$ jet productions,
as pointed theoretically by M.~Drees and R.~M.~Godbole\cite{DREESA}.
It was first confirmed experimentally by AMY\cite{AMY}
at TRISTAN.  In the recent studies done by TOPAZ\cite{TOPAZ93B,HAYASHII}
and ALEPH\cite{ALEPH93} for the two-photon 
process in $e^+e^-$ collisions and those by
H1\cite{H1} and ZEUS\cite{ZEUS} for 
$ep$ collisions, the resolved photon processes are shown to 
constitute the indispensable part of the jet production.
However, comparison of the data and theoretical predictions
are rather qualitative.  
The more precise understanding of the resolved process
is necessary, because it would allow us to test perturbative 
QCD\cite{QCDTEST}
and would provide us valuable information for designing future
$e^+e^-$ and $\gamma\gamma$ 
colliders\cite{MIYAMOTO92,LCBACKGROUND}.

The program uses the program package BASES/SPRING\cite{BASES86,GRACE92}
to calculate the cross section and subsequently generate 
the four momenta of hard jets and remnant jets.
Produced partons are then hadronized using the 
LUND 6.3\cite{LUND63} program 
package in the SPRING step.
The origin of the $\gamma$ can be either bremsstrahlung
or beamstrahlung, according to the parameters of the program.
%We use programs in PYTHIA\cite{PYTHIA56}
%for a  calculation of a parton density functions.

As Monte Calro event generators
for jet production by 
photon-photon collisions in $e^+e^-$ reactions,
PYTHIA\cite{PYTHIA56}
and HERWIG\cite{HERWIG} are also available.
They can include the $\gamma\gamma$ initial state to the existing
general purpose generator.  
The program presented here and the default parameter sets
correspond to those used by TOPAZ group\cite{TOPAZ93B,HAYASHII}.
The ALEPH\cite{ALEPH93} group reported that our program provides better 
descriptions of jets production in two-photon 
process than those of PYTHIA and HERWIG.


In the following sections, we will discuss the 
methods to calculate the cross section and
to generate Monte Calro events.


%%%%%%%%%%%%%%%%%%%%%%%%%%%%%%%%%%%%%%%%%%%
%
% A Method of Monte Calro event simulation.
%
%%%%%%%%%%%%%%%%%%%%%%%%%%%%%%%%%%%%%%%%%%%


\section{A method for Monte Calro simulation}
\label{SECMETHOD}
We use the program package BASES/SPRING\cite{BASES86,GRACE92} to
calculate the cross section and generate four-momentum of final
particles. What we have to prepare when we use the package
is a function to give a differential cross section.

The general expression for the differential cross section 
of jet production by photon-photon collisions in $e^+e^-$ reactions
is given as follows
according to M.~Drees and R.~M.~Godbole\cite{DREESB};
\begin{eqnarray}
 {d\sigma \over dx_1 dx_2 dx_3 dx_4 dcos\hat{\theta}} = 
 f_{\gamma/e^-}(Q^2, x_1) 
         f_{\gamma/e^+}(Q^2, x_2)  
		       D_{p/\gamma}(Q^2,x_3)
	        D_{p/\gamma}(Q^2,x_4) 
	        {d\hat{\sigma}(\hat{s},\hat{t})\over dcos\hat{\theta}},
\end{eqnarray}
where $f_{\gamma/e^\pm}$ are the photon flux factors,
$D_{p/\gamma}$ are the parton ( gluon
or quark ) density functions inside $\gamma$, and
$d\hat{\sigma}(\hat{s},\hat{t})\over dcos\hat{\theta}$ is a differential
cross section of the sub-process. $\hat{\theta}$ is the
scattering angle of the sub-process in its rest.
$x_1$ and $x_2$ are photon energies scaled by beam
energy($E_{beam}$) and $x_3$ and $x_4$ are parton energies scaled by
the original photon energy. 
$\hat{s}\equiv x_1 x_2 x_3 x_4 s$ is the squared invariant energy of the 
sub-processes with $s=4 E^2_{beam}$.
Both $D_{p/\gamma}$ are replaced by $\delta(1-x_i)$ $(i=3\ {\rm and} \ 4)$
in the case of the direct process and
either $D_{p/\gamma}(Q^2,x_3)$ or $D_{p/\gamma}(Q^2,x_4)$
is replaced with $\delta(1-x_i)$ in the case of the once-resolved process.  

In this program,  $f_{\gamma/e^\pm}$ is calculated by the subroutine
{\tt MNJGAM}, $D_{p/\gamma}$ by {\tt MNJPRB}, and 
$d\hat{\sigma}(\hat{s},\hat{t})\over dcos\hat{\theta}$ by {\tt MNJCRS}.
Using these routines, a function {\tt FNCMNJ} gives
${d\sigma \over dx_1 dx_2 dx_3 dx_4 dcos\hat{\theta}}$ for BASES 
calculation.  Number of dimensions in the BASES integration
is six, five of which are used for $x_1$, $x_2$, $x_3$, $x_4$,
and $\cos\hat{\theta}$, while the sixth dimension is used
to select relevant sub-processes.


\subsection{Photon spectrum function}
Two types of bremsstrahlung photon spectrum functions and one type
of beamstrahlung photon spectrum functions are implemented
in the program, which can be selected by variables {\tt NGMINS}
and {\tt NGPLUS} stored in the common {\tt /MNJPRM/},
separately for $e^-$ and $e^+$ beams.

The default bremsstrahlung photon spectrum function is
given by a equivalent photon approximation including 
the non-leading term\cite{BKT71};
\begin{eqnarray}
f_{\gamma/e}(x) &=& {\alpha_{em} \over \pi x }
  \left[ (1 + (1-x)^2) \left\{ \ln ( {E_{beam} \over m_e } ) - 
  \frac{1}{2} \right\}
  + \frac{x^2}{2} \left\{ \ln ( {2(1-x) \over x} ) + 1 \right\} 
  \right. \nonumber \\
&+& \left. \frac{(2-x)^2}{2}\ln ( {2(1-x) \over 2-x} ) \right] 
\label{EQ_EPA}
\end{eqnarray}
\noindent where $E_{beam}$ is the electron(positron) beam energy,
$m_e$ is the electron mass.  We take $\alpha_{em}=1/137.0359895$ throughout 
the program.

For the bremsstrahlung photon spectrum for high energy
$e^\pm$, M.~Drees {\it et al.}\cite{DREESB} suggested the  
formula depend on photon virtuallity with additional suppression
factor, which is given by
\begin{equation}
f_{\gamma/e}(Q^2, x)=0.85{\alpha_{em}\over 2\pi}
  {1+(1-x)^2 \over x }
  \ln({Q^2\over m_e^2} ). \label{EQ_EPAMOD} 
\end{equation}
The value  of $Q$ is set equal to the transverse momentum
$p_t$ of hard jet
or the half of center-of-mass energy of sub-process ($\frac{1}{2}\sqrt{\hat{s}}$)
depending on the flag {\tt NQSSRC} in the common {\tt /MNJPRM/}. 

The beamstrahlung photon spectrum is a function of 
the beam size at interaction point ($\sigma_x$, $\sigma_y$,
and $\sigma_z$), the initial beam energy ($E_{beam}$) and
the number of particle per bunch$(N$). 
Using these parameters, the photon spectrum is given by\cite{BEAMSPECT},
\begin{eqnarray}
f_{\gamma/e}(x) &=&
{\kappa^{1/3}\over \Gamma(1/3)} x^{-2/3} (1-x)^{-1/3} 
   e^{-{\kappa x \over 1 - x}} \nonumber \\
&\times& \left[
{1-\frac{1}{6\sqrt{\kappa}} \over g}
(1 - { 1 - e^{-g N_\gamma} \over g N_\gamma } )
 + \frac{1}{6\sqrt{\kappa}}
(1 - { 1 - e^{-N_\gamma} \over N_\gamma } ), \right] \label{EQ_FLATBEAM}
\end{eqnarray}
\noindent where, 
\begin{eqnarray}
x&=& E_\gamma/E_{beam} ,\\
g&=& 1 - \left[ \frac{N_{CL}}{2N_\gamma} (1+x) 
   + \frac{1}{2} (1-x) \right] (1-x)^{2/3}, \\
\kappa&=& \left(\frac{3}{2} \bar{\Gamma} \right) ^{-1},
\end{eqnarray}
\noindent and
\begin{eqnarray}
\bar{\Upsilon}&=& { 5\gamma_e^2\gamma_0 N \over 
  6 \alpha \sigma_z (\sigma_x + \sigma_y) }, \\
N_{CL} &=& 1.06 \alpha\gamma_e N {2 \over {\sigma_x + \sigma_y}} ,\\
N_\gamma &=& N_{CL} { 1 \over \sqrt{ 1 + \bar{\Upsilon}^{2/3} }}. 
\end{eqnarray}
$\gamma_0$ is a gamma factor of the beam particle 
($\gamma_0=E_{beam}/m_e c^2$) and $\gamma_e$ is a classical electron radius.

This function reproduces the exact spectrum obtained by the 
ABEL simulation\cite{ABEL}, except the slight difference in low photon
energy region\cite{MIYAMOTO92}.  As default parameters,
we prepared the parameter sets for the proposed $e^+e^-$ linear
collider, JLC-I\cite{JLCI}.


\subsection{Parton density function inside the photon}

Due to lack of experimental knowledge,
exact form of parton density inside the photon
is not known well.  It is a
subject of extensive physical investigation.
Several parametrization are available in the program.

DO parameterization\cite{DO} is an asymptotic solution to
photon's APE equation in leading order QCD.
The analytical forms are given by,
\begin{eqnarray}
xD_{q/\gamma}(Q^2,x)&=&F\left[ e_q^2 
{(1.81-1.67x + 2.16x^2)x^{0.70} \over 1 - 0.4\ln{(1-x)}}
+ 0.0038(1-x)^{1.82}x^{-1.18} \right]  \\
xD_{g/\gamma}(Q^2,x)&=&0.194F(1-x)^{1.03}x^{-0.97}
\end{eqnarray}
\noindent where,
\begin{eqnarray}
F={\alpha_{em} \over 2\pi} \ln ( {Q^2 \over \Lambda^2} ).
\end{eqnarray}
%\noindent where $F=(\alpha/2\pi x)\ln(Q^2/\Lambda^2)$.
DO parameterization needs to be supplemented by VDM term
to reproduce data of photon structure function.
The form chosen by Drees and Godbole\cite{DREESA} is,
\begin{eqnarray}
D^{VDM}_{q/\gamma}(Q^2,x)&=&
\alpha_{em}\left[ 5/16{1-x \over \sqrt{x}} + 1/10 {(1-x)^5 \over x} \right] \\
D^{VDM}_{g/\gamma}(Q^2,x)&=&\alpha {2(1-x)^3 \over x}
\end{eqnarray}
We call this as ``DO+VDM'' parameterization. 
%At low $x$, VDM term diverges worser than the $x^{-1.6}$ behaviour
%of the exact asymptotic prediction, M.~Drees {\it et al.} suggested
%the ``modified DO+VMD'' parameterization, where $x$ dependance of
%VDM term is fixed to $x^{-1.6}$ behaviour;
%\begin{equation}
%D_{\gamma/q}(Q^2,x)&=&
%\alpha\left[ 5/16{1-x \over \sqrt{x}} + 1/10 {(1-x)^5 \over x} \right] \\

DG parametrization\cite{DG} uses the APE
equation to determine $Q^2$ evolution of the density
function.
TASSO's data of the photon structure
function at $Q^2=5.9$ GeV$^2$ is used to provide the input
distribution at $Q^2 = 1$ GeV$^2$.
H.~Abramowicz, et al.\cite{LAC}
made the new DG like analysis using the
much larger data sample.  They obtained three solutions
depending on the $Q_0^2$ values and the treatment of the
gluon function.   Their results for gluon density function
are quite different from DG parametrization, especially at low x region.

More recently, M.~Gl\"{u}ck, {\it et al.} determined photonic parton
distribution including higher order  by imposing a VDM valencelike
structure at a low resolution scale\cite{GRV92}.  

%\underline{What about GS and DO.}

In this program package, we include 
DO, DO+VMD and  DG parameterization.
The program of DG parameterization was obtained from 
PYTHIA 7.3 and slightly modified to adopt our program package.
The parton density function is calculated by the subroutine
{\tt MNJPRB} and the function can be selected by its argument,
{\tt NTYP}, which is also accessible by the variable 
{\tt NDISTR} stored in the common
{\tt /MNJPRM/}.  
Other structure functions such as 
LAC are not supplied.
They are provided by PDFLIB\cite{PDFLIB}
and they are implemented easily with 
a slight modification of the subroutine {\tt MNJPRB}.


\subsection{Sub-process cross sections}
\label{SECSUBPROCESS}
The cross section formula for the
sub-processes are categorized into 12 different processes:
(1) $\gamma\gamma\rightarrow q_i \bar{q_i}$,
(2) $\gamma q_i \rightarrow g q_i$,
(3) $\gamma g \rightarrow q_i \bar{q_i}$,
(4) $q_i + q_i \rightarrow q_i + q_i$,
(5) $q_i + q_j \rightarrow q_i + q_j$,
(6) $q_i + \bar{q_i} \rightarrow q_i + \bar{q_i}$,
(7) $q_i + \bar{q_i} \rightarrow q_j + \bar{q_j}$,
(8) $q_i + \bar{q_j} \rightarrow q_i + \bar{q_j}$,
(9) $q_i + \bar{q_i} \rightarrow g + g$,
(10) $q_i + g \rightarrow q_i + g$,
(11) $g + g \rightarrow q_i + \bar{q_i}$
and
(12) $g + g \rightarrow g + g $,
where suffixes indicate quark flavors.
The sub-process (1) is the direct process,
(2) and (3) are the once-resolved processes
and the rest are the twice resolved processes.
We use sixth BASES varaible, $X(6)$, as an index
to the sub-processes. The function {\tt FNCMNJ}
returns the just the differential cross section of given 
process and the total cross section is obtained by BASES
as a part of the integration.
As an option, it is also possible to calculate the cross section of each
sub-processes separetly by setting the parameters {\tt NGNTYP} and {\tt NGNPRC}
in the common {\tt /MNJPRM/} properly.

As initial state quarks, we consider only $u$, $d$ and $s$ quarks
since no heavy quarks components are seen in the photon structure
functions. As final states quarks, we include charm and bottom quarks
in addition to light quarks. 
The cross section formula
used in the program are summarized in the Appendix A.
As a default ({\tt NMASQ=1}),
quark mass effects are included only for charm and bottom quarks,
and massless quark formula are applied for light quarks.
Thresholds for heavy quark productions are set by {\tt QHMASS(1:6)}
in the common {\tt /MNJQMS/} and is also controlled by {\tt NUMFLV} in
the common {\tt /MNJPRM/}.  When we calculate the sub-process cross 
section, strong coupling constant $\alpha_S$ is calculated as,
\begin{equation}
\alpha_S(Q^2)={12 \pi \over (33 - 2 N_f) \log(Q^2/{\Lambda^{(4)}}^2) }
\end{equation}
\noindent where $N_f$ is a number of flavors above threshold.
To allow continuous change of $\alpha_S$ at flavor thresholds,
$\Lambda^{(4)}$ is adjusted when $N_f$ changes to 3 or 5, if 
{\tt NLAM} is set to 1 (not a default).  



%%%%%%%%%%%%%%%%%%%%%%%%%%%%%%%%%%%%
\section{Event Generation}
Once the cross sections are evaluated by BASES, generation  of 
hard scattered partons are straight forward 
with a help of SPRING.  Four momentum of scattered  hard partons
are calculated from the integral variables, $x$.  We do not generate
scattered electrons nor positrons, 
while we generate remnant particles in the case of once-resolved
and twice resolved processes.  
When photon resolved into 
a quark, anti-quark, or gluon, we generate 
an anti-quark, quark or gluon, respectively, along
the direction of beam particle.
Their energy is determined so that energy sum
of the resolved parton and the remnant is equal to 
that of the photon. 
Finally, produced partons are hadronized by using
JETSET 6.3\cite{LUND63} string fragmentation routines.
To use this hadronization program, we needed a trick to treat remnant
jets in the subprocess
$g g \rightarrow g g$, in which case we generate $u$ quarks as remnant
particles instead of gluons.
No parton shower option in JETSET program is used.

The properties of generated events were studied by
TOPAZ\cite{TOPAZ93B}.  According to this study,
particles flows along the jets and particles
activities due to the remnant jets are reasonably 
well described by this event generator.


%%%%%%%%%%%%%%%%%%%%%%%%%%%%%%%%%%%%
%
% Program description.
%
%%%%%%%%%%%%%%%%%%%%%%%%%%%%%%%%%%%%%

\section{Program description}
\subsection{Flow of the job}
As mentioned in the previous section,
we use the program packages BASES/SPRING\cite{BASES86,GRACE92},
which requires two job steps, namely the integration step 
and generation step.  

The JCL example for a job is attached in Appendix~\ref{JCLEXAMPLE}.
The integration step requires four input cards for BASES.
All physical parameters for MINIJET calculations are
prepared in the commons {\tt /MNJPRM/, /MNJQMS/, {\rm and } /MNJBEM/}
which are initialized by the block data {\tt MNJBLK}.
To modify these parameters, user can directly modify 
{\tt MNJBLK} or implement initialization procedure 
in the subroutine {\tt MNJUIN} which is called from 
{\tt USERIN} of BASES. User may initialize 
histogram in the {\tt MNJUIN}.

At the end of BASES, histograms, scatter plots and the total cross
section for given input parameters are printed and the information
for grids and weights of integrations are written into logical 
unit 23 for for the latter use by SPRING.

There are two input cards for generation step, SPRING; number of
events to be generated and CPU time limits.  
It also reads the integration information
from logical unit 23,
which is used to generate events.
The subroutine {\tt MNJMPL} and {\tt MNJHAD}
called from the subroutine {\tt SPEVNT}
are used to convert the integration variables, $x(1:6)$,
into parton four momentum and hadronize, respectively.  
The generated particle information are
stored in the common {\tt /LUJETS/}
with a standard JETSET 6.3 format.  User can access to the common
by the subroutine {\tt MNJUSP} which is called at the end of
every events.


%\begin{picture}(100,200)(10,20)
%\put(1,1){\oval(20,12)}
%\end{picture}

%%%%%%%%%%%%%%%%%
%
%%%%%%%%%%%%%%%%
\subsection{Switches for event type selection}
In this section, we describe the commons which are used to store
physical parameters controlling jet events.
Variables whose name starts by {\tt N} are {\tt INTEGER*4} type, the
others {\tt REAL*8}. Values indicated by D in a parenthesis
are default values which are initialized in the block data {\tt
MNJBLK}.  User can use the subroutine {\tt MNJUIN} to 
set their prefered values.
(R) indicates READ ONLY variables and user should not alter them. 

\newlength{\MNJPRM}
\settowidth{\MNJPRM}{\tt   xCOMMON /MNJPRM/ NGNTYP, NGNPRC, NQSSRC, NUMFLV, XXLAM, XLAM,}
\begin{verbatim}
    COMMON /MNJPRM/ NGNTYP, NGNPRC, NQSSRC, NUMFLV, XXLAM, XLAM,
   >                XXLAM2, XLAM2, YMAX3, YMAX4, PTMIN, PTMAX,
   >                NDISTR, NMASQ, NLAM, NGMINS, NGPLUS, NONEJT
\end{verbatim}
\vspace{-56pt}
\hspace*{3ex}\fbox{\rule[42pt]{\MNJPRM}{0cm}}
\begin{list}{ }{\parsep=0pt \itemsep=0pt \topsep=0pt }
\item[\bf Purpose :] switches and parameters to control generated event
type.  
\item[\bf \tt NGNTYP :] (D=0) select the process type.
  \begin{list}{}{\itemsep=0pt \parsep=0pt \topsep=0pt}
  \item[= 0 :] generate all type of processes.
  \item[= 1 :] generate only once resolved process.
  \item[= 2 :] generate only twice resolved process.
  \item[= 3 :] generate a sub-process specified by {\tt NGNPRC}.
  \end{list}

\item[\bf \tt NGNPRC :] (D=1) select the generated 
sub-process ID (1 to 12),  valid only when {\tt NGNTYP}=3.  
See section~\ref{SECSUBPROCESS} for the meanings of sub-process ID.

\item[\bf \tt NQSSRC :] (D=1) a method to calculate $Q^2$
  \begin{list}{}{\itemsep=0pt \parsep=0pt \topsep=0pt}
  \item[= 0 :] use center-of-mass energy of hard process ($\hat{s}$)
as $Q^2$.
  \item[= 1 :] use $p_t^2$ of hard process as $Q^2$.
  \end{list}

\item[\bf \tt NUMFLV :] (D=4) number of flavors to be generated.
  \begin{list}{}{\itemsep=0pt \parsep=0pt \topsep=0pt}
  \item[= 0 :] number of generated flavors depends on $Q^2$;
$u$, $d$, $s$, $c$, and $b$ quarks when $Q^2 > 500$ GeV/c$^2$, $u$, $d$, $s$,
and $c$ quarks when $50 < Q^2 < 500$ GeV/c$^2$, otherwise $u$, $d$,
and $s$ quarks are generated.
  \item[= 3 :] generate $u$, $d$, and $s$ quarks.
  \item[= 4 :] generate $u$, $d$, $s$ and $c$ quarks.
  \item[= 5 :] generate $u$, $d$, $s$, $c$, and $b$ quarks.
  \end{list}

\item[\bf \tt XXLAM :] (D=0.4 (GeV)) $\Lambda^{(4)}$ used to
calculate  $\alpha_S$.

\item[\bf \tt XLAM :] (D=0.4 (GeV)) $\Lambda^{(4)}$ used to
calculate  photon structure function.

\item[\bf \tt XXLAM2,XLAM2 :] (R) internal use only
({\tt XXLAM2=XXLAM$^2$, XLAM2=XLAM$^2$}).

\item[\bf \tt NLAM :] (D=0) treatment of $\Lambda^{(4)}$ in hard
scattering ({\tt XXLAM}).
  \begin{list}{}{\itemsep=0pt \parsep=0pt \topsep=0pt}
  \item[= 0 :] use fixed value as given by {\tt XXLAM} independent 
  of number of generated flavors ($N_f$).
  \item[= 1 :] {\tt XXLAM} is used when $N_f$=4, but changed when $N_f$=3 
or 5 so that $\alpha_s$ changes continuously at the threshold of 
new flavors.
  \end{list}

\item[\bf \tt NDISTR :] (D=1) Type of photon structure function.
  \begin{list}{}{\itemsep=0pt \parsep=0pt \topsep=0pt}
  \item[= 0 :] use DG structure function\cite{DG}. Number of flavor
is changed automatically as the case of {\tt NUMFLV=0}.
  \item[= 1 :] same as {\tt NDISTR}=0 case  but {\tt NUMFLV} is fixed
to 3.
  \item[= 2 :] same as {\tt NDISTR}=0 case  but {\tt NUMFLV} is fixed to 4.
  \item[= 3 :] same as {\tt NDISTR}=0 case  but {\tt NUMFLV} is fixed to 5.
  \item[= 4 :] reserved for LAC\cite{LAC} parameterization.
  \item[= 5 :] reserved for LAC\cite{LAC} parameterization.
  \item[= 6 :] reserved for LAC\cite{LAC} parameterization.
  \item[= 7 :] use DO\cite{DO} structure function.
  \item[= 8 :] use a sum of VMD and DO structure function.
  \item[= 9 :] use a sum of modified VMD and DO structure function.
  \end{list}

\item[\bf \tt NMASQ :] (D=1) select the method  of quark treatment in
hard processes.
  \begin{list}{}{\itemsep=0pt \parsep=0pt \topsep=0pt}
  \item[= 0 :] all quark are assumed to be massless.
  \item[= 1 :] take into account $c$ and $b$ quark mass effect.
  \item[= 2 :] take into account the mass effect for all quarks, but
the formula used is correct only for direct process.
  \end{list}

\item[\bf \tt NGMINS :] (D=0) photon spectrum function from $e^-$ beam.
  \begin{list}{}{\itemsep=0pt \parsep=0pt \topsep=0pt}
  \item[= 0 :] bremsstrahlung photon spectrum by EPA as given in 
Eq.\ref{EQ_EPA}. 
  \item[= 1 :] bremsstrahlung photon spectrum by modified EPA forumula
as given in Eq.\ref{EQ_EPAMOD}.
%  \item[= 10 :] beamstrahlung photon spectrum of linear collider with
%a round beam optics(Eq.\ref{EQ_FLATBEAM}.
  \item[= 21 - 26 :] beamstrahlung photon spectrum of linear collider with
a flat beam optics, whose parameter is given by an array 
{\tt (BEMAMP(i,{\tt NGMINS-20}),i=1,7)} in a common {\tt /MNJBEM/}.
  \end{list}

\item[\bf \tt NGPLUS :] (D=0) same as {\tt NGMINS}, but for $e^+$
beam.

\item[\bf \tt YMAX3 :] (D=0.7) maximum absolute value of rapidity 
for one of two jets produced in hard processes.

\item[\bf \tt YMAX4 :] (D=0.7) maximum absolute value of rapidity 
for the other jet.  If {\tt YMAX4} is negative, no cut on rapidity is
applied to this jet and the cross section is counted twice if {\tt
NONEJT} = 2 and the absolute values of the rapidities of both hard
jets are less than {\tt YMAX3}.

\item[\bf \tt PTMIN :] (D=3.0 (GeV)) minimum $p_t$ of hard jet.

\item[\bf \tt PTMAX :] (D=8.0 (GeV)) maximum $p_t$ of hard jet.

\item[\bf \tt NONEJT :] (D=1) A way to count cross section, when
calculate one-jet inclusive cross section, and valid when {\tt YMAX4} 
is negative.
  \begin{list}{}{\itemsep=0pt \parsep=0pt \topsep=0pt}
  \item[= 1 :] always count one-jet as one jet. 
  \item[= 2 :] count one-jet twice if the absolute values of
the rapidity of both jets are less than {\tt YMAX3}.
  \end{list}

\end{list}

%%%%%%%%%%%%%%%%%%%%%%%%%%%%%%%%%%
%
%  common /MNJQMS/
%
%%%%%%%%%%%%%%%%%%%%%%%%%%%%%%%%%%

\newlength{\MNJQMS}
\settowidth{\MNJQMS}{\tt   COMMON /MNJQMS/ QMAS(6), QHMAS(6)}
\begin{verbatim}
    COMMON /MNJQMS/ QMAS(6), QHMAS(6)
\end{verbatim}
\vspace{-28pt}
\hspace*{4ex}\fbox{\rule[14pt]{\MNJQMS}{0cm}}
\begin{list}{ }{\parsep=0pt \itemsep=0pt \topsep=0pt }
\item[\bf Purpose :] quark masses ({\tt QMAS(I)}) and 
thresholds ({\tt QHMAS(I)}) for corresponding flavor productions.
These are {\tt REAL*8} variables.  This common is used in
the subroutine {\tt MNJCRS} and {\tt MNJMPL}.
\item[\tt QMAS(1) = :] (D=0.325 GeV) $d$ quark mass.
\item[\tt QMAS(2) = :] (D=0.325 GeV) $u$ quark mass.
\item[\tt QMAS(3) = :] (D=0.5 GeV) $s$ quark mass.
\item[\tt QMAS(4) = :] (D=1.5 GeV) $c$ quark mass.
\item[\tt QMAS(5) = :] (D=4.5 GeV) $b$ quark mass.
%\item[\tt QMAS(6) = :] not used.
\item[\tt QHMAS(1) = :] (D=0.325 GeV) threshold for $d$ quark generation.
\item[\tt QHMAS(2) = :] (D=0.325 GeV) threshold for $u$ quark generation.
\item[\tt QHMAS(3) = :] (D=0.5 GeV) threshold for $s$ quark generation.
\item[\tt QHMAS(4) = :] (D=1.87 GeV) threshold for $c$ quark generation.
\item[\tt QHMAS(5) = :] (D=5.3 GeV) threshold for $b$ quark generation.
%\item[\tt QHMAS(6) = :] not used.
\end{list}


%%%%%%%%%%%%%%%%%%%%%%%%%%%%%%%%%%
%
%  common /MNJBEM/
%
%%%%%%%%%%%%%%%%%%%%%%%%%%%%%%%%%%
\newlength{\MNJBEM}
\settowidth{\MNJBEM}{\tt   COMMON /MNJBEM/ BEAMPR(7,6)}
\begin{verbatim}
    COMMON /MNJBEM/ BEAMPR(7,6)
\end{verbatim}
\vspace{-28pt}
\hspace*{4ex}\fbox{\rule[14pt]{\MNJBEM}{0cm}}
\begin{list}{ }{\parsep=0pt \itemsep=0pt \topsep=0pt }
\item[\bf Purpose :] Beam parameters of linear colliders for
beamstrahlung photon generation,
which is used in the subroutine {\tt MNJGAM}. {\tt BEAMPR} is 
a {\tt REAL*8} array.  The first index corresponds to the parameters as
described below, while the second index corresponds to the different
machine optics.  As a default, JLC-I\cite{JLCI} parameters are
stored; {\tt BEAMPR(1:7,1)}=S-band at $\sqrt{s}=300$ GeV,
{\tt BEAMPR(1:7,2)}=S-band at $\sqrt{s}=500$ GeV,
{\tt BEAMPR(1:7,3)}=L-band at $\sqrt{s}=300$ GeV,
{\tt BEAMPR(1:7,4)}=L-band at $\sqrt{s}=500$ GeV,
{\tt BEAMPR(1:7,5)}=X-band at $\sqrt{s}=300$ GeV,
{\tt BEAMPR(1:7,6)}=X-band at $\sqrt{s}=500$ GeV.
\item[\tt BEAMPR(1,1:6) = :] beam energy, $E_0$ (GeV).
\item[\tt BEAMPR(2,1:6) = :] beam spot size in x direction 
at interaction point (IP), $\sigma_x$ (cm).
\item[\tt BEAMPR(3,1:6) = :] beam spot size in y at IP, $\sigma_y$ (cm).
\item[\tt BEAMPR(4,1:6) = :] beam spot size in y at IP, $\sigma_z$ (cm).
\item[\tt BEAMPR(5,1:6) = :] Number of particles in a bunch.
\item[\tt BEAMPR(6,1:6) = :] Luminosity without  pinch effect.
\item[\tt BEAMPR(7,1:6) = :] Number of beamstrahlung photons.
\end{list}

%%%%%%%%%%%%%%%%%%%%%%%%%%%%%%%%%%
%
%  common /MNJEVT/
%
%%%%%%%%%%%%%%%%%%%%%%%%%%%%%%%%%%
\subsection{Event information}
Since we use JETSET 6.3 to hadronize partons, four momentum of final
state particles are stored in the common {\tt /LUJETS/},
whose format are described in Ref.~\cite{LUND63}.
User can use the subroutine {\tt MNJUSP} to access {\tt /LUJETS/}
which are called at the end of the event generation in SPRING.
Kinematical informations used by BASES/SPRING are stored in the
common {\tt /MNJEVT/}, which is described below.
Variables whose name starts by {\tt N} are {\tt INTEGER*4} type, the
others {\tt REAL*8}. 
All variables other than {\tt EBEAM} are READ ONLY.
{\tt EBEAM} is electron/positron beam energy,
whose default value is 29 GeV.

\newlength{\MNJEVT}
\settowidth{\MNJEVT}{\tt x COMMON /MNJEVT/ RS, EBEAM, XG(25), EGAM1, EGAM2, CS, PCM,}
\begin{verbatim}
    COMMON /MNJEVT/ RS, EBEAM, XG(25), EGAM1, EGAM2, CS, PCM,
   >                EGAM10, EGAM20, FONE, NPRC, INDX
\end{verbatim}
\vspace{-42pt}
\hspace*{3ex}\fbox{\rule[28pt]{\MNJEVT}{0cm}}
\begin{list}{ }{\parsep=0pt \itemsep=0pt \topsep=0pt }
\item[\bf Purpose :] contains the generated event information. 
\item[\bf \tt RS :] (R (GeV)) center of mass energy.
\item[\bf \tt EBEAM :] (D=29.0 GeV) beam energy, which should be set
by user in the block data {\tt MNJBLK} or the subroutine {\tt MNJUIN}.
\item[\bf \tt XG(1:6) :] (R) a copy of random variables used for the
event generation ( see section~\ref{SECMETHOD} for description. ).
\item[\bf \tt XG(21) :] (R) the scaled energy of photon emitted from
$e^-$ beam ($x_1 \equiv E_{\gamma} / E_{beam}$).
\item[\bf \tt XG(22) :] (R) the scaled energy of photon emitted from
$e^+$ beam ($x_2 \equiv E_{\gamma} / E_{beam}$).
\item[\bf \tt XG(23) :] (R) the scaled energy of partons resolved
from photon emitted by  $e^-$ ($x_3 \equiv E_{parton} / E_{\gamma}$).
\item[\bf \tt XG(24) :] (R) the scaled energy of partons resolved
from photon emitted by $e^+$ ($x_4 \equiv E_{parton} / E_{\gamma}$).
\item[\bf \tt  CS :] (R) $\cos\theta$ of hard jets in its
center-of-mass system.
\item[\bf \tt  PCM :] (R) momentum of hard jets.
\item[\bf \tt EGAM10 :] (R (GeV) ) photon energy (GeV) emitted from
$e^-$ 
\item[\bf \tt EGAM20 :] (R (GeV)) photon energy (GeV) emitted
from $e^+$ 
\item[\bf \tt EGAM1 :] (R (GeV)) energy of a parton in the initial
state of hard processes.
\item[\bf \tt EGAM2 :] (R (GeV)) energy of the other parton in the initial
state of hard processes.
\item[\bf \tt FONE :] (R) weight factor of events, valid when
calculating one jet inclusive cross section.
\item[\bf \tt NPRC :] (R) process ID of generated event.
\item[\bf \tt INDX :] (R) pointer to the array {\tt KCDATA}, which
gives the process type and parton IDs involved in hard process.

\end{list}

%%%%%%%%%%%%%%%%%%%%%%%%%%%%%%%%%%%%%%%%%%%%%%%%%%%%%%%%%%%%%%%%%%
%
%  List of subroutines.
%
%%%%%%%%%%%%%%%%%%%%%%%%%%%%%%%%%%%%%%%%%%%%%%%%%%%%%%%%%%%%%%%%%%

\subsection{Breif description of subroutines}
Purpose and arguments of the subroutines are described.
First parameter (I or O) in the parenthisis of the arguments discriptions
are the flag to indicate 
whether it is input or output arguments.  Second parameter
is a varaible type. 
\newlength{\MNJCRS}
\settowidth{\MNJCRS}{\tt xSUBROUTINE MNJCRS( NPROC, SHAT, IAD1, IAD2, IAD3, IAD4,}
\begin{verbatim} 
    SUBROUTINE MNJCRS( NPROC, SHAT, IAD1, IAD2, IAD3, IAD4,
   >           COSTH, PTMIN, NMASQ,    DSDCS , PCM, QSQ)
\end{verbatim}
\vspace*{-42pt}
\hspace*{3ex}\fbox{\rule[28pt]{\MNJCRS}{0cm}}
\begin{list}{ }{\parsep=0pt \itemsep=0pt \topsep=0pt }
\item[\bf Purpose :] calculate a cross section of sub-processes.
\item[\tt NPROC :] (I,{\tt I*4}) sub-process ID as described in the 
subsection \ref{SECSUBPROCESS}. 
Negative {\tt NPROC} means the process where the initial particles
are interchanged.
\item[\tt SHAT :] (I, {\tt R*8}) squared center-of-mass energy of the sub-process.
\item[\tt  IAD1, IAD2, IAD3, IAD4 :] (I, {I*4}) particles IDs involved
in the sub-processes; {\tt IAD1, IAD2} for initial particles,
and {\tt IAD3, IAD4} for final particles. Particle assignments to the
IDs are same as those used by PDG group\cite{PDG}.  
\item[\tt COSTH :] (I, {\tt R*8}) production angle in center-of-mass
system of the sub-process.
\item[\tt PTMIN :] (I, {\tt R*8}) minimum $p_t$ requirement for final particles.
\item[\tt  NMASQ :] (I, {\tt I*4}) treatment of quark mass effect.
Same meanings as those stored in the common {\tt /MNJPRM/}.
\item[\tt DSDCS :] (O, {\tt R*8}) differential cross section 
($d\sigma/d\cos\hat{\theta}$) for the sub-process in the unit of pb.
\item[\tt PCM :] (O, {\tt R*8}) momentum of a final state particle.
\item[\tt QSQ :] (O, {\tt R*8}) energy scale, $Q^2$, after take into
account the quark mass effect.
\end{list}

\newlength{\MNJTYP}
\settowidth{\MNJTYP}{\tt    SUBROUTINE MNJTYP( NTYPE, X, EBEAM, FX )}
{ \samepage
\begin{verbatim}
    SUBROUTINE MNJTYP( NTYPE, X, EBEAM, FX )
\end{verbatim}
\vspace*{-28pt}
\hspace*{4ex}\fbox{\rule[14pt]{\MNJTYP}{0cm}}}
\begin{list}{ }{\parsep=0pt \itemsep=0pt \topsep=0pt }
\item[\bf Purpose :] get the weight of a photon.
\item[\tt NTYPE :] (I, {\tt I*4}) type of photon spectrum.  
See the description of the common {\tt /MNJPRM/}.
\item[\tt X :] (I, {\tt R*8}) scaled photon energy, 
$X = E_{\gamma}/E_{beam}$.
\item[\tt EBEAM :] (I, {\tt R*8}) beam energy.
\item[\tt FX :] (O, {\tt R*8}) weight of the photon.
\end{list}


%\begin{verbatim}
% SUBROUTINE USERIN
%\end{verbatim}
%\begin{list}{ }{\parsep=0pt \itemsep=0pt \topsep=0pt }
%\item[\bf Purpose :] called by BASES program and do initialization
%for numerical integration.
%\end{list}

\newlength{\FNCMNJ}
\settowidth{\FNCMNJ}{\tt REAL FUNCTION FNCMNJ*8(X)}
\begin{verbatim}
    REAL FUNCTION FNCMNJ*8(X)
\end{verbatim}
\vspace*{-28pt}
\hspace*{4ex}\fbox{\rule[14pt]{\FNCMNJ}{0cm}}
\begin{list}{ }{\parsep=0pt \itemsep=0pt \topsep=0pt }
\item[\bf Purpose :] called by BASES program and calculate the
cross section for the phase space given by {\tt X}.
\end{list}


%\begin{verbatim}
% SUBROUTINE USROUT
%\end{verbatim}
%\begin{list}{ }{\parsep=0pt \itemsep=0pt \topsep=0pt }
%\item[\bf Purpose :] called by BASES program at the end of 
%the integration and printout the results.
%\end{list}


\newlength{\MNJHAD}
\settowidth{\MNJHAD}{\tt SUBROUTINE MNJHAD(NPART, RBUF)}
\begin{verbatim}
    SUBROUTINE MNJHAD(NPART, RBUF)
\end{verbatim}
\vspace{-28pt}
\hspace*{4ex}\fbox{\rule[14pt]{\MNJHAD}{0cm}}
\begin{list}{ }{\parsep=0pt \itemsep=0pt \topsep=0pt }
\item[\bf Purpose :] stores particle information generated by SPRING
into the common {\tt /LUJETS/} and hadronize them 
by using JETSET6.3 routines.
\item[\tt NPART :] (I, {\tt I*4}) number of particles.
\item[\tt RBUF(20,NPART) :] (I, {\tt I*4})
particle information generated by SPRING. The array elements not
listed below are reserved for future use.
  \begin{list}{ }{\parsep=0pt \itemsep=0pt \topsep=0pt }
  \item[\tt  RBUF( 1,I) =:] particle serial number.
  \item[\tt RBUF( 2,I) =:] particle ID according to the PDG convension\cite{PDG}.
  \item[\tt RBUF( 3,I) =:] particle mass (GeV).
  \item[\tt RBUF( 4,I) =:] charge.
  \item[\tt RBUF( 5,I) =:] $P_x$ (GeV).
  \item[\tt RBUF( 6,I) =:] $P_y$ (GeV).
  \item[\tt RBUF( 7,I) =:] $P_z$ (GeV).
  \item[\tt RBUF( 8,I) =:] Energy GeV).
  \item[\tt RBUF(12,I) =:] number of daughter particles.
  \item[\tt RBUF(13,I) =:] particle serial number of the first daughter.
  \item[\tt RBUF(14,I) =:] particle serial number of the last daughter.
  \item[\tt RBUF(20,I) =:] set to 0 when particle is at the end of the
color string, otherwise set to 10000.
  \end{list}
\end{list}

\newlength{\MNJMPL}
\settowidth{\MNJMPL}{\tt SUBROUTINE MNJMPL(NFINAL, PARTON, GBUF, PVCT)}
\begin{verbatim}
    SUBROUTINE MNJMPL(NFINAL, PARTON, GBUF, PVCT)
\end{verbatim}
\vspace{-28pt}
\hspace*{4ex}\fbox{\rule[14pt]{\MNJMPL}{0cm}}
\begin{list}{ }{\parsep=0pt \itemsep=0pt \topsep=0pt }
\item[\bf Purpose :] using the four-momentum information generated by
SRING, prepares the array, {\tt PARTON}, for the hadronization 
by the subroutine {\tt MNJHAD}.
\item[\tt NFINAL :] (O, {\tt I*4}) number of final particles.
\item[\tt PARTON(20,4) :] (O, {\tt R*4}) final particle information.
Contents are same as those stored in the argument array {\tt RBUF} of the 
subroutine {\tt MNJHAD}.
\item[\tt GBUF(20) :] (O, {\tt R*4}) general event information,
prepared for documentation purpose.
  \begin{list}{ }{\parsep=0pt \itemsep=0pt \topsep=0pt }
  \item[\tt GBUF( 1) =:] energy of resolved parton from $e^-$ side 
(={\tt EGAM1}).
  \item[\tt GBUF( 2) =:] energy of resolved parton from $e^+$ side 
(={\tt EGAM2}).
  \item[\tt GBUF( 3) =:] energy of photon from $e^-$ (={\tt EGAM10}).
  \item[\tt GBUF( 4) =:] energy of photon from $e^+$ (={\tt EGAM20}).
  \item[\tt GBUF( 5) =:] center-of-mass energy of hard process
($\hat{s}$).
  \item[\tt GBUF( 6) =:] $p_t$ of final particles of sub-process.
  \item[\tt GBUF( 7) =:] $\cos\hat{\theta}$ of the sub-process.
  \item[\tt GBUF( 8) =:] $\hat{\theta}$ of the sub-process.
  \item[\tt GBUF( 9) =:] center-of-mass energy of the $\gamma\gamma$ system.
  \item[\tt GBUF(10) =:] rapidity of first parton in laboratory system.
  \item[\tt GBUF(11) =:] rapidity of the other parton in the
laboratory system.
  \item[\tt GBUF(12) =:] momentum of final particle produced in hard process.
  \item[\tt GBUF(20) =:] weight factor of a event.
  \end{list}  
\item[\tt PVCT(4,3) :] (O, {\tt R*4})four-momentum array, {\tt PVCT(1,I)}=$p_x$,
 {\tt PVCT(2,I)}=$p_y$, {\tt PVCT(3,I)}=$p_z$,
and  {\tt PVCT(4,I)}=energy.
  \begin{list}{ }{\parsep=0pt \itemsep=0pt \topsep=0pt }
  \item[ \tt PVCT(J,1) =:] four-momentum of center-of-mass system of
sub-process in laboratory system.
  \item[ \tt PVCT(J,2) =:] four-momentum of first final state particle in
the center-of-mass system of the sub-process.
  \item[ \tt PVCT(J,3) =:] four-momentum of the other final state particle in
the center-of-mass system of the sub-process.
  \end{list}
\end{list}

{\samepage
\newlength{\MNJPRB}
\settowidth{\MNJPRB}{\tt SUBROUTINE MNJPRB(NTYP, NFLV, QSQ, X, XLAM2, PROB)}
\begin{verbatim}
    SUBROUTINE MNJPRB(NTYP, NFLV, QSQ, X, XLAM2, PROB)
\end{verbatim}
\vspace{-28pt}
\hspace*{4ex}\fbox{\rule[14pt]{\MNJPRB}{0cm}}}
\begin{list}{ }{\parsep=0pt \itemsep=0pt \topsep=0pt }
\item[\bf Purpose :] calculate the parton density inside the photon.
\item[ \tt NTYP :] (I, {\tt I*4}) 
select the type of photon structure function.
See the description of the common {\tt /MNJPRM/} for the meanings.
\item[ \tt NFLV :] (I, {\tt I*4}) particle ID resolved from photon,
1=$d$ quark, 2=$u$, 3=$s$, c=$c$, 5=$b$, and 21=gluon.  
\item[ \tt QSQ :] (I, {\tt R*8}) energy scale.
\item[ \tt X :] (I, ({\tt R*8}) scaled energy.
\item[ \tt XLAM2 :] (I, {\tt R*8}) $\Lambda^2$.
\item[ \tt PROB :] (O, {\tt R*8}) the value of the parton density in
the photon.
\end{list}


%\begin{verbatim}
% SUBROUTINE SPINIT
%\end{verbatim}
%\begin{list}{ }{\parsep=0pt \itemsep=0pt \topsep=0pt }
%\item[\bf Purpose :] initialization routine for SPRING step. 
%The seed for random variable and optional parameters for JETSET 6.3
%should be initialized here.
%\end{list}

%\begin{verbatim}
% SUBROUTINE SPEVNT
%\end{verbatim}
%\begin{list}{ }{\parsep=0pt \itemsep=0pt \topsep=0pt }
%\item[\bf Purpose :] called by SPRING and generate one event.
%\end{list}


\newlength{\MNJSPE}
\settowidth{\MNJSPE}{\tt     SUBROUTINE MNJSPE}
\begin{verbatim}
    SUBROUTINE MNJSPE
\end{verbatim}
\vspace{-28pt}
\hspace*{4ex}\fbox{\rule[14pt]{\MNJSPE}{0cm}}
\begin{list}{ }{\parsep=0pt \itemsep=0pt \topsep=0pt }
\item[\bf Purpose :] called at the end of SPRING step and print out
seed of a random variable.
\end{list}

%\begin{verbatim}
% SUBROUTINE MNJTIT
%\end{verbatim}
%\begin{list}{ }{\parsep=0pt \itemsep=0pt \topsep=0pt }
%\item[\bf Purpose :] print parameters.
%seed.
%\end{list}


\newlength{\MNJUIN}
\settowidth{\MNJUIN}{\tt     SUBROUTINE MNJUIN}
\begin{verbatim}
    SUBROUTINE MNJUIN
\end{verbatim}
\vspace{-28pt}
\hspace*{4ex}\fbox{\rule[14pt]{\MNJUIN}{0cm}}
\begin{list}{ }{\parsep=0pt \itemsep=0pt \topsep=0pt }
\item[\bf Purpose :] 
called from {\tt USERIN} and initialize
parameters and histograms required for each calculations.
\end{list}


\newlength{\MNJUFL}
\settowidth{\MNJUFL}{\tt         SUBROUTINE MNJUFL}
\begin{verbatim}
    SUBROUTINE MNJUFL
\end{verbatim}
\vspace{-28pt}
\hspace*{4ex}\fbox{\rule[14pt]{\MNJUFL}{0cm}}
\begin{list}{ }{\parsep=0pt \itemsep=0pt \topsep=0pt }
\item[\bf Purpose :] 
called from {\tt FNCMNJ} to accumulate the histogram data.
\end{list}



%\begin{verbatim}
% SUBROUTINE DRNSET(ISEED)
%\end{verbatim}
%\begin{list}{ }{\parsep=0pt \itemsep=0pt \topsep=0pt }
%\item[\bf Purpose :] Initialize a seed for random number generation.
%Prepared here to override the same routine in BASES/SPRING so as to 
%allow the re-generation of the event which has been suspended
%previously in the middle of the generation.
%\item[\tt ISEED :] (I,{\tt I*4}) seed for random number generation.
%\end{list}


%\begin{verbatim}
% REAL FUNCTION DRN*8(ISEED)
%\end{verbatim}
%\begin{list}{ }{\parsep=0pt \itemsep=0pt \topsep=0pt }
%\item[\bf Purpose :] generate uniform random number between 0 to 1
%using the function {\tt RLU} prepared in JETSET6.3\cite{LUND63}.
%\item[\tt ISEED :] not used.
%\end{list}


%\begin{verbatim}
% SUBROUTINE DRNGET(ISEED)
%\end{verbatim}
%\begin{list}{ }{\parsep=0pt \itemsep=0pt \topsep=0pt }
%\item[\bf Purpose :] get a seed of random number.
%\item[\tt ISEED :] (O,{\tt I*4})seed of the random number.
%\end{list}

%\begin{verbatim}
% SUBROUTINE LRNSET(ISEED)
%\end{verbatim}
%\begin{list}{ }{\parsep=0pt \itemsep=0pt \topsep=0pt }
%\item[\bf Purpose :] set the seed of random number in to 
%the common {\tt /LUSEED/} for the use by JETSET6.3 program.
%\item[\tt ISEED :] (I,{\tt I*4})seed of the random number.
%\end{list}


%\begin{verbatim}
% SUBROUTINE LRNGET(ISEED)
%\end{verbatim}
%\begin{list}{ }{\parsep=0pt \itemsep=0pt \topsep=0pt }
%\item[\bf Purpose :] get the seed of random number from 
%the common {\tt /LUSEED/}.
%\item[\tt ISEED :] (O,{\tt I*4})seed of the random number.
%\end{list}

\newlength{\MNJBLK}
\settowidth{\MNJBLK}{\tt     BLOCK DATA MNJBLK}
\begin{verbatim}
    BLOCK DATA MNJBLK
\end{verbatim}
\vspace{-28pt}
\hspace*{4ex}\fbox{\rule[14pt]{\MNJBLK}{0cm}}
\begin{list}{ }{\parsep=0pt \itemsep=0pt \topsep=0pt }
\item[\bf Purpose :] block data, where the commons
{\tt /MNJPRM/, /MNJBEM/, /MNJQMS/} are initialized.
\end{list}

%\begin{verbatim}
% SUBROUTINE UBSTBK( PB, PR, PA)
%\end{verbatim}
%\begin{list}{ }{\parsep=0pt \itemsep=0pt \topsep=0pt }
%\item[\bf Purpose :] utility routine to make a Lorentz transformation 
%to four-momentum {\tt PB(4)} in {\tt PR(4)}-rest frame to
%{\tt PA(4)} in {\tt PR(4)} moving system. 
%\end{list}


%\begin{verbatim}
% SUBROUTINE UBSTFD( PB, PR, PA)
%\end{verbatim}
%\begin{list}{ }{\parsep=0pt \itemsep=0pt \topsep=0pt }
%\item[\bf Purpose :] utility routine to make a Lorentz transformation 
%to four-momentum {\tt PB(4)} to {\tt PA(4)}
%in {\tt PR(4)} rest frame.
%\end{list}

%\begin{verbatim}
% SUBROUTINE UVZERO( NWORDS, ARRAY )
%\end{verbatim}
%\begin{list}{ }{\parsep=0pt \itemsep=0pt \topsep=0pt }
%\item[\bf Purpose :] utility routine to clear 
%{\tt NWORDS } words of {\tt REAL*4} array {\tt ARRAY}.
%\end{list}
%%\end{document}

\section{Conclusion}
We have developed the event generation code for jets production
in two-photon processes in $e^+e^-$ reactions.
The program can generate direct, once-resolved and twice-resolved
processes separately or simultaneously.
The program was tested by TOPAZ and ALEPH 
experiments and
successful to reproduce event properties of
inclusive jet production and hadronic event topologies


\vskip 1cm
\leftline{\bf Acknowledgment}
We are grateful to S.~Kawabata and Minami-Tateya group at KEK for
help in using BASES/SPRING system.
This work was made in the course of the experimental study of $e^+e^-$
data taken by the TOPAZ group and design studies of a future $e^+e^-$
linear collider, JLC.  
We thank members of the projects for useful discussions and encouragements.



\vskip 1cm
\begin{center}
\bf REFERENCES
\end{center}

%%%%%%%%%%%%%%%%%%%%%%%%%%%%%%%
%
%  Bibliography
%
%%%%%%%%%%%%%%%%%%%%%%%%%%%%
\begin{thebibliography}{99}


% 
% "Probing the hadronic structure of the photon at TRISTAN"
\bibitem{DREESA}
M.~Drees and R.~M.~Godbole, Nucl.~Phys.~B339(1990)355.

%
%
\bibitem{AMY}
R.~Tanaka {\elevenit  et al.}, (AMY Collab.),
{\elevenit Phys. Lett.} {\elevenbf B277} (1992) 215.

%
% ``Measurement of the Inclusive Cross Section of Jets in \gamma\gamma
% Interactions at TRISTAN.''
\bibitem{TOPAZ93B}
H.~Hayashii {\elevenit et al.}, (TOPAZ Collab.), 
{\elevenit Phys. Lett.} {\elevenbf B314} (1993) 149.

%
\bibitem{HAYASHII}
H.~Hayashii, NWU-HEP 92-03, June 1992. To appear in the proceeding
of the 9-th International Workshop on Photon-Photon collisions,
San Diego, USA, March, 1992;
H.~Hayashii, talk presented at Monion Conference.
%
	
%
% ``An Experimental Study of \gamma\gamma\rightarrow hadrons at LEP''
%
\bibitem{ALEPH93}
D.~Buskulic {\elevenit et al.}, (ALEPH Collab.),
{\elevenit CERN-PPE/93-94}, {\elevenit to be appear in Phys. Lett. B}.

%
%
\bibitem{H1}
T.~Ahmed {\elevenit et al.}, (H1 Collab.),
{\elevenit Phys. Lett.} {\elevenbf B297} (1992) 205.

%
\bibitem{ZEUS}
M.~Derrick {\elevenit et al.}, (ZEUS Collab.),
{\elevenit Phys. Lett.} {\elevenbf B297} (1992) 404.

\bibitem{BASES86}
S.~Kawabata, {\elevenit Comp. Phys. Comm.}
{\elevenbf 41} (1986) 127.

\bibitem{GRACE92}
T.~Ishikawa {\elevenit  et al.}, {\elevenit KEK Report 92-19},
February 1993.

\bibitem{QCDTEST}
for example,
R.~Enomoto {\elevenit et al.}, (TOPAZ Collab.,)
{\elevenit KEK Preprint 93-107}, August, 1993;
M.~Drees, M.~Kramer, J.~Zunft, and P.~M.~Zerwas,
{\elevenit Phys. Lett. } {\elevenbf B306} (1993) 371.

%
% ``Minijet background at JLC''
\bibitem{MIYAMOTO92}
A.~Miyamoto, in the {\elevenit Proceedings of the Third Workshop
on Japan Linear Collider}, ed. A.~Miyamoto, 
{\elevenit KEK Proceedings 92-13}, December, 1992, p.389.

%
% ``Minijet background at JLC''
%\bibitem{MIYAMOTOFFIR}

\bibitem{LCBACKGROUND}
M.~Drees and R.~M.~Godbole,
{\elevenit DESY 92-044}, March, 1992;
P.~Chen, T.~L.~Barklow  and M.~E.~Peskin, 
{\elevenit SLAC-PUB-5873}, April, 1993;
A.~Miyamoto, in the {\elevenit Proceedings of JLC-FFIR92}, 
ed. T.~Tauchi and N.~Yamamoto, 
{\elevenit KEK Proceedings 93-6}, June 1993, p. 541.

%
\bibitem{LUND63}
T.~Sj\"{o}strand and M.~Bengtsson, Comp. Phys. Comm. 43(1987)367.
%We used LUND6.3 default hadronization parameters.
%

%
% ``PYTHIA5.6 and JETSET 7.3 ; Physics and Manual''
\bibitem{PYTHIA56}
H.~-U.~Bengtsson and T.~Sj\"{o}strand,
{\elevenit Comp. Phys. Comm.} {\elevenbf 46} (1987) 43;
T.~Sj\"{o}strand, {\elevenit CERN-TH.6488/92}, May, 1992. 

%
% ``HERWIG Monte Calro''
\bibitem{HERWIG}
G.~Marchesini, B.~R.~Webber, {\elevenit Nucl. Phys.}
{\elevenbf B310} (1988) 461.

%
\bibitem{DREESB}
M.~Drees and R.~M.~Godbole, 
{\elevenit Phys. Rev. Lett.} {\elevenbf  67} (1991) 1189.
%

%
% EPA formula with including non-leading term
\bibitem{BKT71}
S.~Brodsky, T.~Kinoshita and H.~Terasawa,
{\elevenit Phys. Rev.} {\elevenbf D41} (1971) 1532.


\bibitem{BEAMSPECT}
K.~Yokoya, private communication. see also
P.~Chen, SLAC-PUB-5615.
%

\bibitem{ABEL}
K.~Yokoya, ABEL, KEK-Report-85-9, October. 1985, 
also Nucl. Instr. Meth. B251 (1986) 1.
%

%
\bibitem{JLCI}
JLC Group, {\elevenit KEK Report 92-16}, December, 1992.


%
%   DO parametrization.
\bibitem{DO}
D.~W.~Duke and J.~F.~Owens, Phys.~Rev.~D26(1982)1600.
%
%   DG
\bibitem{DG}
M.~Drees and K.~Grassie, Z.~Phys.~C28(1985)451.
%
%  LAC
% "Minijets and large hadronic backgrounds at e+e- supercolliders"
\bibitem{LAC}
H.~Abramowicz, K.~Charchula and A.~Levy, 
Phys. Lett. B269(1991)458.
%
%


%
\bibitem{GRV92}
M.~Gl\"{u}ck, E.~Reya, and A.~Vogt,
{\elevenit Phys. Rev.} {\elevenbf D46} (1992) 1973.

\bibitem{PDFLIB}
H.~Plothow-Besch, 
{\elevenit Comp. Phys. Comm.} {\elevenbf 75} (1993) 396.

%
\bibitem{PDG}
Particle Data Group, 
{\elevenit Phys. Rev.} {\elevenbf D45} (1992) S1.

%
%\bibitem{HARDCROS}
%J.~F.~Owens and E.~Reya, 
%{\elevenit Phys. Rev.} {\elevenbf 18} (1978) 1501;
%B.~L.~Combridge, J.~Kripfganz and J.~Ranft,
%{\elevenit Phys. Lett.} {\elevenbf 70B} (1977) 234;
%D.~W.~Duke and J.~F.~Owens,
%{\elevenit Phys. Rev.} {\elevenbf D26} (1982) 1600.
%
\bibitem{HARDCROS}
%\bibitem{DO82}
D.~W.~Duke and J.~F.~Owens,
{\elevenit Phys. Rev.} {\elevenbf D26} (1982) 1600;
%
%\bibitem{JW78}
L.~M.~Jones and H.~W.~Wyld,
{\elevenit Phys. Rev.} {\elevenbf D17} (1978) 759;
%
%\bibitem{CKR77}
B.~L.~Combridge, J.~Kripfganz and J.~Ranft,
{\elevenit Phys. Lett.} {\elevenbf 70B} (1977) 234;
J.~F.~Owens, E.~Reya and M.~Gl\"{u}ck,
{\elevenit Phys. Rev.} {\elevenbf D18} (1978) 1501;
%
%\bibitem{GOR78}
B.~L.~Combridge,
{\elevenit Nucl. Phys.} {\elevenbf B151} (1979) 429;
M.~Gl\"{u}ck, J.~F.~Owens and E.~Reya,
{\elevenit Phys. Rev.} {\elevenbf D17} (1978) 2324.




%
% "Probing the hadronic structure of the photon at TRISTAN"
%\bibitem{TWOPHOTONBOOK}
%H.~Kolanoski and P.~Zerwas, "High Energy Electron-Positron Physics",
%eds. A.~Ali and P.~S\"{o}ding, World Scientific, pp. 695.
%
%\bibitem{TWOGAMEXP}
%CELLO Collaboration, Z. Phys. C 51(1991)365;
%TPC/Two-Gamma Collaboration, Phys. Rev. D41(1990)2667;
%PLUTO Collaboration, Phys. Lett. B149(1984)421; Z. Phys. C 26(1984)191;
%Z. Phys. C 33(1987)351.
%
%\bibitem{EIKONAL}
%J.~R.~Forshaw and J.~K.~Storrow, Phys. Lett. B268(1991)116; Phys. Lett.
%B278(1992)193.
%
%\bibitem{BKGREF}
%M.~Drees and R.~M.~Godbole, DESY 92-044, March, 1992.
%
% "Differential luminosity under multiphoton beamstrahlung"
%\bibitem{PSCHENA}
%P.~Chen, SLAC-PUB-5615.

%\bibitem{YOKOYAA}
%K.~Yokoya, talk presented at Workshop on Physics and Experiments with
%Linear Colliders, Saariselk\"{a}, Finland, Sept. 1991. 

\end{thebibliography}

%
%%%%%%%%%%%%%%%%%%%%%%%%%%%%%%%%%%%%%%%%%%%%%%%%
%
%  Appendix
%
%%%%%%%%%%%%%%%%%%%%%%%%%%%%%%%%%%%%%%%%%%%%%%%%

\appendix



\section{The cross section formula for the sub-processes}

For completeness, we summarize below the cross section formula 
for the sub-processes used in the program\cite{HARDCROS}.
In the formula below, $\hat{\theta}$ is a scattering angle in 
the center-of-mass system and $\hat{s}$, $\hat{t}$ and $\hat{u}$ are
mandelstarm variables.  The suffixes for $q$ indicate the quark
flavour,
$e_q$ is a quark charge, and $\hat{\beta}$ is a quark velocity: 
$\hat{\beta}=\sqrt{1-4m^2_q/\hat{s}}$.
The formula eq.\ref{EQA4} and \ref{EQA12} do not include
the factors of 2 in the original paper as we integrate
all polar angle range.  


%We use the massive quark formula for the processes
%which may involve charm quark, namely the processes (A-1), (A-3),
%(A-7) and (A-10).  Otherwise, mass less quark formula 
%were used. 
%
%$\alpha_S$ is calculated by a formula,
%\begin{eqnarray}
%\alpha_S(Q^2)={12\pi \over (33-2N_F)\log(Q^2/\Lambda^2) }
%\end{eqnarray}.


\subsection{$\gamma\gamma \rightarrow q_i\bar{q_i}$ }
\begin{eqnarray}
{d\sigma \over d\cos\hat{\theta}} &=&
{3\pi\alpha^2 e_q^4 \hat{\beta} \over \hat{s} } \left[
\frac{m^2_q -\hat{t}}{m^2_q-\hat{u}}  +
 \frac{m^2_q -\hat{u}}{m^2_q-\hat{t}} +
 4 \left( \frac{m^2_q}{m^2_q-\hat{t}} + \frac{m^2_q}{m^2_q-\hat{u}}  
  \right) \right. \nonumber \\
&-& \left. 4  \left( \frac{m^2_q}{m^2_q-\hat{t}} +
\frac{m^2_q}{m^2_q-\hat{u}}  
  \right)^2 \right].
\end{eqnarray}
%\noindent where $\hat{\beta}=\sqrt{1-\frac{4m^2_q}{\hat{s}}}$.

%\subsection{$\gamma q_i \rightarrow g q_i $}
%\subsection{$\gamma q_i \rightarrow q_i g $ \cite{DO82}}
\subsection{$\gamma q_i \rightarrow q_i g $ }
\begin{equation}
{d\sigma \over d\cos\hat{\theta}} = - { 4\pi\alpha e_q^2 \alpha_S \over 3\hat{s}}
( \frac{\hat{t}}{\hat{s}} + \frac{\hat{s}}{\hat{t}} )
\end{equation}

%\subsection{$\gamma g \rightarrow q_i\bar{q_i}$\cite{JW78}}
\subsection{$\gamma g \rightarrow q_i\bar{q_i}$}
\begin{eqnarray}
{d\sigma \over d\cos\hat{\theta}} &=&
{\pi\alpha e_q^2 \alpha_S\hat{\beta} \over 2\hat{s} } \left[
\frac{m^2_q -\hat{t}}{m^2_q-\hat{u}}  +
 \frac{m^2_q -\hat{u}}{m^2_q-\hat{t}} +
 4 \left( \frac{m^2_q}{m^2_q-\hat{t}} + \frac{m^2_q}{m^2_q-\hat{u}}  
  \right) \right. \nonumber \\
&-& \left. 4  \left( \frac{m^2_q}{m^2_q-\hat{t}} +
\frac{m^2_q}{m^2_q-\hat{u}}  
  \right)^2 \right].
\end{eqnarray}
% \noindent where $\hat{\beta}=\sqrt{1-\frac{4m^2_q}{\hat{s}}}$.

%\subsection{$q_i + q_i \rightarrow q_i + q_i$\cite{CKR77}}
\subsection{$q_i + q_i \rightarrow q_i + q_i$}
\begin{eqnarray}
{d\sigma \over d\cos\hat{\theta}} &=&
{\pi\alpha_S^2 \over 4\hat{s} } 
\left[ \frac{4}{9} \left( 
{\hat{s}^2 + \hat{u}^2 \over \hat{t}^2} + 
{\hat{s}^2 + \hat{t}^2 \over \hat{u}^2} \right) 
- \frac{8}{27}{\hat{s}^2 \over \hat{u}\hat{t}} \right] \label{EQA4} 
\end{eqnarray}

%\subsection{$q_i + q_j \rightarrow q_i + q_j$\cite{CKR77}}
\subsection{$q_i + q_j \rightarrow q_i + q_j$}
\begin{eqnarray}
{d\sigma \over d\cos\hat{\theta}} &=&
{\pi\alpha_S^2 \over 2\hat{s} } 
\left[ \frac{4}{9} {\hat{s}^2 + \hat{u}^2 \over \hat{t}^2} \right]
\end{eqnarray}

%\subsection{$q_i + \bar{q}_i \rightarrow q_i + \bar{q}_i$\cite{CKR77}}
\subsection{$q_i + \bar{q}_i \rightarrow q_i + \bar{q}_i$}
\begin{eqnarray}
{d\sigma \over d\cos\hat{\theta}} &=&
{\pi\alpha_S^2 \over 2\hat{s} } 
\left[ \frac{4}{9} \left( 
{\hat{s}^2 + \hat{u}^2 \over \hat{t}^2} + 
{\hat{t}^2 + \hat{u}^2 \over \hat{s}^2} \right) 
- \frac{8}{27}{\hat{u}^2 \over \hat{s}\hat{t}} \right] 
\end{eqnarray}


%\subsection{$q_i + \bar{q}_i \rightarrow q_j + \bar{q}_j$\cite{GOR78}}
\subsection{$q_i + \bar{q}_i \rightarrow q_j + \bar{q}_j$}
\begin{eqnarray} 
{d\sigma \over d\cos\hat{\theta}} &=&
{\pi\alpha_S^2\hat{\beta} \over 2\hat{s} } 
\frac{4}{9} \left[ { (m^2_q -\hat{t})^2 + 
(m^2_q-\hat{u})^2 + 2m^2_q\hat{s} \over \hat{s}^2 } \right]
\end{eqnarray}
% \noindent where $\hat{\beta}=\sqrt{1-\frac{4m^2_q}{\hat{s}}}$.

%\subsection{$q_i + \bar{q}_j \rightarrow q_i + \bar{q}_j$\cite{CKR77}}
\subsection{$q_i + \bar{q}_j \rightarrow q_i + \bar{q}_j$}
\begin{eqnarray} 
{d\sigma \over d\cos\hat{\theta}} &=&
{\pi\alpha_S^2 \over 2\hat{s} } 
\frac{4}{9} \left[ {\hat{s}^2 + \hat{u}^2 \over \hat{t}^2 } \right]
\end{eqnarray}

%\subsection{$q_i + \bar{q}_i \rightarrow g + g$\cite{CKR77}}
\subsection{$q_i + \bar{q}_i \rightarrow g + g$}
\begin{eqnarray} 
{d\sigma \over d\cos\hat{\theta}} &=&
{\pi\alpha_S^2 \over 2\hat{s} } 
 \left[ \frac{32}{27}{\hat{u}^2 + \hat{t}^2 \over \hat{u}\hat{t} }
- \frac{8}{3}{\hat{u}^2 + \hat{t}^2 \over \hat{s}^2 } \right]
\end{eqnarray}

%\subsection{$q_i + g \rightarrow q_i + g$\cite{CKR77}}
\subsection{$q_i + g \rightarrow q_i + g$}
\begin{eqnarray} 
{d\sigma \over d\cos\hat{\theta}} &=&
{\pi\alpha_S^2 \over 2\hat{s} } 
 \left[ -\frac{4}{9}{\hat{u}^2 + \hat{s}^2 \over \hat{u}\hat{s} }
+{\hat{u}^2 + \hat{s}^2 \over \hat{t}^2 } \right]
\end{eqnarray}


%\subsection{$g + g \rightarrow q_i + \bar{q}_i$\cite{GOR78}}
\subsection{$g + g \rightarrow q_i + \bar{q}_i$}
\begin{eqnarray} 
{d\sigma \over d\cos\hat{\theta}} &=&
{\pi\alpha_S^2\hat{\beta} \over 32\hat{s} } 
 \left[ 12 {(m_q^2-\hat{t})(m_q^2-\hat{u}) \over \hat{s}^2 }
+ \frac{8}{3} { (m_q^2-\hat{t})(m_q^2-\hat{u}) - 2m_q^2(m_q^2+\hat{t})
   \over (m_q^2 - \hat{t})^2 }
\right. \nonumber \\
&+& \frac{8}{3} { (m_q^2-\hat{t})(m_q^2-\hat{u}) - 2m_q^2(m_q^2+\hat{u})
   \over (m_q^2 - \hat{u})^2 }
- \frac{2}{3}{m_q^2(\hat{s}-4m_q^2) 
  \over (m_q^2-\hat{t})(m_q^2-\hat{u}) }
\nonumber \\ 
&-&  6 { (m_q^2-\hat{t})(m_q^2-\hat{u}) + m_q^2(\hat{u}-\hat{t})
   \over \hat{s}(m_q^2-\hat{t}) }
\nonumber \\
&-& \left. 6 { (m_q^2-\hat{t})(m_q^2-\hat{u}) + m_q^2(\hat{t}-\hat{u})
   \over \hat{s}(m_q^2-\hat{u}) } \right]
\end{eqnarray}


%\subsection{$g + g \rightarrow g + g $\cite{GOR78}}
\subsection{$g + g \rightarrow g + g $}
\begin{eqnarray} 
{d\sigma \over d\cos\hat{\theta}} &=&
{\pi\alpha_S^2 \over 4\hat{s} } 
 \frac{9}{2}\left[ 3 - {\hat{u}\hat{t} \over \hat{s}^2 }
- {\hat{u}\hat{s} \over \hat{t}^2 } 
- {\hat{s}\hat{t} \over \hat{u}^2 } \right] \label{EQA12}
\end{eqnarray}

\section{Example of JCL on the FACOM machine at KEK}
\label{JCLEXAMPLE}
\begin{verbatim}
//job-ID JOB CLASS=S
//*
//*  BASES step
//*
// EXEC FORT7CLG,
//      PARM.FORT='OPT(3),NOS,NOSTATIS',
//      PARM.LKED='NOMAP,LET'
//FORT.SYSINC DD DSN=user-ID.MINIJET.FORT,DISP=SHR
//FORT.SYSIN  DD DSN=user-ID.MINIJET.FORT(BSMAIN),DISP=SHR
//LKED.SYSLIB DD
//            DD DSN=user-ID.BASES50.LOAD,DISP=SHR
//*
//GO.SYSIN DD *
     1,  1  Current loop count,   Max. Loop count
    -4      Print Flag
     0      Input Flag
     5.     CPU time limit in minutes
//******   BASES results.
//GO.FT23F001 DD DISP=(NEW,PASS),
// DSN=&&BASES,SPACE=(TRK,(10,10),RLSE)
//*
//*  Spring step.
//*
//  EXEC  FORT7CLG,
//  PARM.FORT='OPT(3),NOS,AE,NOSTATIS',
//  PARM.LKED='NOMAP'
//FORT.SYSINC DD DSN=user-ID.MINIJET.FORT,DISP=SHR
//FORT.SYSIN  DD DSN=user-ID.MINIJET.FORT(MNJSPR),DISP=SHR
//LKED.SYSLIB DD
//            DD DSN=user-ID.BASES50.LOAD,DISP=SHR
//            DD DSN=user-ID.G#LUND63.LOAD,DISP=SHR
//*
//GO.FT05F001 DD *
    10000000
   20.0
//*
//******   BASES results to read
//GO.FT23F001 DD DISP=(OLD,PASS),DSN=&&BASES
//*
\end{verbatim}

\newpage
\section{Test run results}

Test run outputs for $p_t$ distribution of 
direct (ID=13), once-resolved(ID=14) and twice-resolved(ID=15)
processes at $\sqrt{s}=58$ GeV.

\begin{small}
\begin{verbatim}
 Histogram (ID = 13 ) for Pt(for Direct)                                                  
                              Linear Scale indicated by "*"
     x      d(Sigma)/dx     0.0E+00      1.4E+00      2.8E+00     4.1E+00      
 +-------+------------------+------------+------------+-----------+-----------+
 I  E  0 I 0.000        E  0I                                                 I
 I 3.000 I 4.273+-0.026 E  0I***************************************OOOOOOO   I
 I 3.200 I 3.631+-0.020 E  0I**********************************OOOOOOOOOO     I
 I 3.400 I 2.981+-0.015 E  0I****************************OOOOOOOOOOOOO        I
 I 3.600 I 2.527+-0.013 E  0I***********************OOOOOOOOOOOOOOOO          I
 I 3.800 I 2.175+-0.011 E  0I********************OOOOOOOOOOOOOOOOOO           I
 I 4.000 I 1.826+-0.009 E  0I*****************OOOOOOOOOOOOOOOOOOO             I
 I 4.200 I 1.573+-0.008 E  0I***************OOOOOOOOOOOOOOOOOOO               I
 I 4.400 I 1.387+-0.007 E  0I*************OOOOOOOOOOOOOOOOOOO                 I
 I 4.600 I 1.176+-0.006 E  0I***********OOOOOOOOOOOOOOOOOOO                   I
 I 4.800 I 1.041+-0.006 E  0I**********OOOOOOOOOOOOOOOOOOO                    I
 I 5.000 I 9.198+-0.052 E -1I*********OOOOOOOOOOOOOOOOOO                      I
 I 5.200 I 7.994+-0.048 E -1I********OOOOOOOOOOOOOOOOOO                       I
 I 5.400 I 7.149+-0.044 E -1I*******OOOOOOOOOOOOOOOOO                         I
 I 5.600 I 6.342+-0.040 E -1I******OOOOOOOOOOOOOOOOO                          I
 I 5.800 I 5.563+-0.037 E -1I******OOOOOOOOOOOOOOO                            I
 I 6.000 I 5.049+-0.035 E -1I*****OOOOOOOOOOOOOOO                             I
 I 6.200 I 4.498+-0.036 E -1I*****OOOOOOOOOOOOOO                              I
 I 6.400 I 3.989+-0.031 E -1I****OOOOOOOOOOOOO                                I
 I 6.600 I 3.666+-0.029 E -1I****OOOOOOOOOOOO                                 I
 I 6.800 I 3.311+-0.027 E -1I****OOOOOOOOOOO                                  I
 I 7.000 I 2.980+-0.026 E -1I***OOOOOOOOOOO                                   I
 I 7.200 I 2.697+-0.023 E -1I***OOOOOOOOO                                     I
 I 7.400 I 2.469+-0.023 E -1I***OOOOOOOO                                      I
 I 7.600 I 2.210+-0.020 E -1I***OOOOOOO                                       I
 I 7.800 I 2.058+-0.020 E -1I**OOOOOOO                                        I
 I  E  0 I 0.000        E  0I                                                 I
 +-------+------------------+---------------------------+---------------------+
     x      d(Sigma)/dx     1.0E-01                     1.0E+00                
                              Logarithmic Scale indicated by "O"
\end{verbatim}
\newpage
\begin{verbatim}  
 Histogram (ID = 14 ) for Pt(for 1 resolved)                                              
                              Linear Scale indicated by "*"
     x      d(Sigma)/dx     0.0E+00      1.0E+00      2.0E+00     3.0E+00      
 +-------+------------------+------------+------------+-----------+-----------+
 I  E  0 I 0.000        E  0I                                                 I
 I 3.000 I 3.300+-0.036 E  0I******************************************OOOOO  I
 I 3.200 I 2.444+-0.026 E  0I*******************************OOOOOOOOOOOOOO    I
 I 3.400 I 1.935+-0.021 E  0I*************************OOOOOOOOOOOOOOOOOO      I
 I 3.600 I 1.568+-0.016 E  0I********************OOOOOOOOOOOOOOOOOOOOO        I
 I 3.800 I 1.216+-0.013 E  0I****************OOOOOOOOOOOOOOOOOOOOOOO          I
 I 4.000 I 1.014+-0.011 E  0I*************OOOOOOOOOOOOOOOOOOOOOOOOO           I
 I 4.200 I 8.342+-0.091 E -1I***********OOOOOOOOOOOOOOOOOOOOOOOOO             I
 I 4.400 I 6.795+-0.078 E -1I*********OOOOOOOOOOOOOOOOOOOOOOOOO               I
 I 4.600 I 5.609+-0.069 E -1I********OOOOOOOOOOOOOOOOOOOOOOOOO                I
 I 4.800 I 4.621+-0.057 E -1I******OOOOOOOOOOOOOOOOOOOOOOOOO                  I
 I 5.000 I 4.068+-0.051 E -1I******OOOOOOOOOOOOOOOOOOOOOOOO                   I
 I 5.200 I 3.358+-0.045 E -1I*****OOOOOOOOOOOOOOOOOOOOOOOO                    I
 I 5.400 I 2.776+-0.039 E -1I****OOOOOOOOOOOOOOOOOOOOOOO                      I
 I 5.600 I 2.447+-0.036 E -1I****OOOOOOOOOOOOOOOOOOOOOO                       I
 I 5.800 I 2.077+-0.030 E -1I***OOOOOOOOOOOOOOOOOOOOOO                        I
 I 6.000 I 1.846+-0.030 E -1I***OOOOOOOOOOOOOOOOOOOOO                         I
 I 6.200 I 1.540+-0.025 E -1I**OOOOOOOOOOOOOOOOOOOO                           I
 I 6.400 I 1.368+-0.023 E -1I**OOOOOOOOOOOOOOOOOOOO                           I
 I 6.600 I 1.149+-0.020 E -1I**OOOOOOOOOOOOOOOOOO                             I
 I 6.800 I 1.042+-0.019 E -1I**OOOOOOOOOOOOOOOOO                              I
 I 7.000 I 9.116+-0.184 E -2I**OOOOOOOOOOOOOOOO                               I
 I 7.200 I 8.049+-0.158 E -2I**OOOOOOOOOOOOOOO                                I
 I 7.400 I 7.143+-0.144 E -2I*OOOOOOOOOOOOOOO                                 I
 I 7.600 I 6.556+-0.129 E -2I*OOOOOOOOOOOOOOO                                 I
 I 7.800 I 5.782+-0.120 E -2I*OOOOOOOOOOOOOO                                  I
 I  E  0 I 0.000        E  0I                                                 I
 +-------+------------------+------------------+------------------+-----------+
     x      d(Sigma)/dx     1.0E-02            1.0E-01            1.0E+00      
                              Logarithmic Scale indicated by "O"
\end{verbatim}
\newpage
\begin{verbatim}  
 Histogram (ID = 15 ) for Pt(for two resolved)                                            
                              Linear Scale indicated by "*"
     x      d(Sigma)/dx     0.0E+00      1.1E+00      2.3E+00     3.4E+00      
 +-------+------------------+------------+------------+-----------+-----------+
 I  E  0 I 0.000        E  0I                                                 I
 I 3.000 I 3.458+-0.053 E  0I***************************************OOOOOOOOO I
 I 3.200 I 2.380+-0.037 E  0I***************************OOOOOOOOOOOOOOOOOO    I
 I 3.400 I 1.728+-0.028 E  0I********************OOOOOOOOOOOOOOOOOOOOOO       I
 I 3.600 I 1.250+-0.020 E  0I**************OOOOOOOOOOOOOOOOOOOOOOOOO          I
 I 3.800 I 9.584+-0.170 E -1I***********OOOOOOOOOOOOOOOOOOOOOOOOOO            I
 I 4.000 I 6.898+-0.122 E -1I********OOOOOOOOOOOOOOOOOOOOOOOOOOO              I
 I 4.200 I 5.279+-0.095 E -1I******OOOOOOOOOOOOOOOOOOOOOOOOOO                 I
 I 4.400 I 4.144+-0.078 E -1I*****OOOOOOOOOOOOOOOOOOOOOOOOO                   I
 I 4.600 I 3.297+-0.066 E -1I****OOOOOOOOOOOOOOOOOOOOOOOOO                    I
 I 4.800 I 2.529+-0.052 E -1I***OOOOOOOOOOOOOOOOOOOOOOO                       I
 I 5.000 I 1.990+-0.040 E -1I***OOOOOOOOOOOOOOOOOOOOOO                        I
 I 5.200 I 1.595+-0.034 E -1I**OOOOOOOOOOOOOOOOOOOOO                          I
 I 5.400 I 1.245+-0.027 E -1I**OOOOOOOOOOOOOOOOOOO                            I
 I 5.600 I 1.083+-0.031 E -1I**OOOOOOOOOOOOOOOOOO                             I
 I 5.800 I 8.496+-0.213 E -2I*OOOOOOOOOOOOOOOOO                               I
 I 6.000 I 7.033+-0.236 E -2I*OOOOOOOOOOOOOOO                                 I
 I 6.200 I 5.995+-0.196 E -2I*OOOOOOOOOOOOOO                                  I
 I 6.400 I 4.599+-0.144 E -2I*OOOOOOOOOOOO                                    I
 I 6.600 I 4.343+-0.137 E -2I*OOOOOOOOOOO                                     I
 I 6.800 I 3.593+-0.170 E -2I*OOOOOOOOOO                                      I
 I 7.000 I 2.881+-0.091 E -2I*OOOOOOOO                                        I
 I 7.200 I 2.612+-0.091 E -2I*OOOOOOO                                         I
 I 7.400 I 2.037+-0.076 E -2I*OOOOO                                           I
 I 7.600 I 1.806+-0.068 E -2I*OOOO                                            I
 I 7.800 I 1.621+-0.067 E -2I*OOO                                             I
 I  E  0 I 0.000        E  0I                                                 I
 +-------+------------------+------------------+------------------+-----------+
     x      d(Sigma)/dx     1.0E-02            1.0E-01            1.0E+00      
                              Logarithmic Scale indicated by "O"
\end{verbatim}
\end{small}

\newpage

\begin{figure}[p]
\vskip 15cm
\caption{ \label{DIAGRAM}
Diagrams contributing to the jet productions
in two-photon processes in $e^+e^-$ reactions;
(a) direct process, (b) once-resolved process,
(c) twice-resolved process.}
\end{figure}



\end{document}


